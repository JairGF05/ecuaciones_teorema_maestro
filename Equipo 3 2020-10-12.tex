%%%%%%%%%%%%%%%%%%%%%%%%%%%%%%%%%%%%%%%%%%%%%%%%%%%%%%%%%%%%%%%%%%%%%%%%%%%
%
% Plantilla para un articulo en LaTeX en español.
%
%%%%%%%%%%%%%%%%%%%%%%%%%%%%%%%%%%%%%%%%%%%%%%%%%%%%%%%%%%%%%%%%%%%%%%%%%%%

\documentclass{article}
\usepackage{listings}
\usepackage{enumerate}
% Esto es para poder escribir acentos directamente:
\usepackage[latin1]{inputenc}
\usepackage{graphicx}
% Esto es para que el LaTeX sepa que el texto esta en espa\~{n}ol:
\usepackage[spanish]{babel}
% Paquetes de la AMS:
\usepackage{amsmath, amsthm, amsfonts}
%Package de alineación
\usepackage[document]{ragged2e}
\spanishdecimal{.}

% Teoremas
%--------------------------------------------------------------------------
\newtheorem{thm}{Teorema}[section]
\newtheorem{cor}[thm]{Corolario}
\newtheorem{lem}[thm]{Lema}
\newtheorem{prop}[thm]{Proposici\'on}
\theoremstyle{definition}
\newtheorem{defn}[thm]{Definici\'on}
\theoremstyle{remark}
\newtheorem{rem}[thm]{Observaci\'on}

% Atajos.
% Se pueden definir comandos nuevos para acortar cosas que se usan
% frecuentemente. Como ejemplo, aqui se definen la R y la Z dobles que
% suelen representar a los conjuntos de numeros reales y enteros.
%--------------------------------------------------------------------------

\def\RR{\mathbb{R}}
\def\ZZ{\mathbb{Z}}

% De la misma forma se pueden definir comandos con argumentos. Por
% ejemplo, aqui definimos un comando para escribir el valor absoluto
% de algo mas facilmente.
%--------------------------------------------------------------------------
\newcommand{\abs}[1]{\left\vert#1\right\vert}

% Operadores.
% Los operadores nuevos deben definirse como tales para que aparezcan
% correctamente. Como ejemplo definimos en jacobiano:
%--------------------------------------------------------------------------
\DeclareMathOperator{\Jac}{Jac}

%--------------------------------------------------------------------------
\title{Actividades sesi\'on 12 de octubre\\
An\'alisis de Algoritmos IV\\
Maestr\'ia en Ciencias de la Computaci\'on\\ 
Oto\~{n}o 2020\\
BUAP
}


\author{Equipo 3\\
  \small Erick Barrios Gonz\'alez\\
  \small Oscar Eduardo Gonz\'alez Ramos\\
  \small Oswaldo Jair Garc\'ia Franco
}

\begin{document}

\maketitle
\clearpage
%\abstract{Analisis de algoritmos}

\section{Ejercicios}
Resuelva las siguientes recurrencias exactamente,perimero mediante cambio de variable,
luego verifique sus respuestas empleando el resultado del ejemplo 4.\\

%inicia ejercicio 1.1

\begin{enumerate}[1]
\item  $$t_n = \left \{ 
\begin{matrix} 
1 & \mbox{si n = }1\\ 
2T(n/2)+1 & \mbox{en}\mbox{ otro caso}\\
\end{matrix}
\right. $$\\


\begin{center}
Reemplazamos $n$ por $2^{i}$: \\
$t_{i}= T(2^{i})$\\
\vspace{.5cm}
$n/2$ se convierte en :\\
$\frac{2^{i}}{2} =2^{i-1} $\\
\vspace{.5cm}
Entonces:\\
$t_{i}= T(2^{i})= 2T(2^{i-1})+1$\\
\hspace{1.6cm}$=2t_{i-1}+1 $\\
\vspace{.5cm}
Se reescribe como:\\
$t_{i}- 2t_{i-1} = 1$\\
\vspace{.5cm}
Que es la forma de una recurrencia no homog\'enea\\
Su polinomio caracter\'istico es:\\
$(x-2)(x-1)$\\
\vspace{.5cm}
As\'i todas las coluciones para $t_{i}$ son de la forma:\\
$t_{i} = c_{1}(2^{i}) + c_{2}$\\
\vspace{.5cm}
Ahora regresamos a la ecuaci\'on original\\
usamos el hecho de que $T(2^{i})=t_{i}$, As\'i $T(n) = tlgn$ \\
$T(n)= c_{1}2^{lgn}+ c2$ $=$ $c_{1}n + c_{2}$ $(1)$\\
\vspace{.5cm}
Lo cual es suficiente para concluir que :\\
$T(n) \in O(n|n$ es potencia de 2)\\
Como queremos obtener el orden exacto debemos probar que $c_{1}$ es
estrictamente positivo.\\
\vspace{.5cm}
Sustituyendo la soluci\'on proporcionada por la ecuaci\'on (1) en la recurencia original.\\
$1 = T(n)-2T(n/2)$\\
\hspace{1.6cm}$= (c_{1}n+c2)- 2(c_{1}\frac{n}{2} + c_{2})$\\
$c_{2}= -1$\\
No obtuvimos el valor de $c_{1}$ pero estamos en condiciones de afirmar que 
debe ser estrictamente positiva, de lo contrario la ecuaci\'on $(1)$ indicar\'ia falsamente que $T(n)$ es negativa.\\
Entonces queda establecido que:\\
$T(n) \in \Theta(n|n$ es potencia de 2)\\

\vspace{.5cm}
Comprobaci\'on con teorema Maestro:
$l = 2$, $b = 2$,$k = 0$\\
$l > 2^{0} = 1$\\
Entonces $T(n) \in \Theta(n^{\log_{2}(2)})$\\
$T(n) \in \Theta(n|n$ es potencia de 2)

\end{center}
\clearpage
%termina ejercicio 1.1

%inicio del ejercicio 1.2
\item  $$t_n = \left \{ 
\begin{matrix} 
1 & \mbox{si n = }1\\ 
4T(n/2)+n & \mbox{en}\mbox{ otro caso}\\
\end{matrix}
\right. $$\\


\begin{center}
Reemplazamos $n$ por $2^{i}$: \\
$t_{i}= T(2^{i})$\\
\vspace{.5cm}
$n/2$ se convierte en :\\
$\frac{2^{i}}{2} =2^{i-1} $\\
\vspace{.5cm}
Entonces:\\
$t_{i}= T(2^{i})= 4T(2^{i-1})+ 2^{i}$\\
\hspace{1.6cm}$=4t_{i-1}+2^{i} $\\
\vspace{.5cm}
Se reescribe como:\\
$t_{i}- 4t_{i-1} = 2^{i}$\\
Que es la forma de una recurrencia no homog\'enea\\
\vspace{.5cm}
Su polinomio caracter\'istico es:\\
$(x-4)(x-2)$\\
\vspace{.5cm}
As\'i todas las coluciones para $t_{i}$ son de la forma:\\
$t_{i} = c_{1}(4^{i}) + c_{2}(2^{i})$\\
\vspace{.5cm}
Ahora regresamos a la ecuaci\'on original\\
usamos el hecho de que $T(2^{i})=t_{i}$, As\'i $T(n) = tlgn$ \\
$T(n)= c_{1}4^{lgn}+ c_{2}2^{lgn}$ $=$ $c_{1}n^{2} + c_{2}n$ $(1)$\\
\vspace{.5cm}
Lo cual es suficiente para concluir que :\\
$T(n) \in O(n^{2}|n$ es potencia de 2)\\
\vspace{.5cm}
Como queremos obtener el orden exacto debemos probar que $c_{1}$ es
estrictamente positivo.\\
Sustituyendo la soluci\'on proporcionada por la ecuaci\'on (1) en la recurencia original.\\
$n = T(n)-4T(n/2)$\\
\hspace{1.6cm}$= (c_{1}n^{lg4} + c_{2}n)- 4(c_{1}(\frac{n}{2})^{lg4} + c_{2}(\frac{n}{2}))$\\
$c_{2}= -1$\\
No obtuvimos el valor de $c_{1}$ pero estamos en condiciones de afirmar que 
debe ser estrictamente positiva, de lo contrario la ecuaci\'on $(1)$ indicar\'ia falsamente
que $T(n)$ es negativa.\\

Entonces queda establecido que:\\
$T(n) \in \Theta(n^{2}|n$ es potencia de 2)\\
\vspace{.5cm}
Comprobaci\'on con teorema Maestro:
$l = 4$, $b = 2$,$k = 0$\\
$l > 2^{0} = 1$\\
Entonces $T(n) \in \Theta(n^{\log_{2}(4)})$\\
$T(n) \in \Theta(n^{2}|n$ es potencia de 2)

\end{center}
\clearpage
%fin ejercicio 1.2

\end{enumerate}


\clearpage

\pagebreak 
% Bibliograf\'ia.
%-----------------------------------------------------------------
\begin{thebibliography}{99}
\bibitem{Mauricio2016} Lavalle Mart\'inez, Jos\'e de Jes\'us; La presentaci\'on sobre la tercera
parte de An\'alisis de algoritmos, An\'alisis y Dise\~{n}o de Algoritmos.
Buap, Oto\~{n}o 2020.

\bibitem {notas} Jim\'enez Salazer, H\'ector y Lavalle Mart\'inez, Jos\'e de Jes\'us; An\'alisis y Dise\~{n}o de Algoritmos. Traducci\'on de partes del libro Fundamentals of Algorithmics de Brassard and Bratley FCC - BUAP, 2020.





\end{thebibliography}

\end{document}