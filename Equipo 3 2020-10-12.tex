%%%%%%%%%%%%%%%%%%%%%%%%%%%%%%%%%%%%%%%%%%%%%%%%%%%%%%%%%%%%%%%%%%%%%%%%%%%
%
% Plantilla para un articulo en LaTeX en español.
%
%%%%%%%%%%%%%%%%%%%%%%%%%%%%%%%%%%%%%%%%%%%%%%%%%%%%%%%%%%%%%%%%%%%%%%%%%%%

\documentclass{article}
\usepackage{listings}
\usepackage{enumerate}
% Esto es para poder escribir acentos directamente:
\usepackage[latin1]{inputenc}
\usepackage{graphicx}
% Esto es para que el LaTeX sepa que el texto esta en espa\~{n}ol:
\usepackage[spanish]{babel}
% Paquetes de la AMS:
\usepackage{amsmath, amsthm, amsfonts}
%Package de alineación
\usepackage[document]{ragged2e}
\spanishdecimal{.}

% Teoremas
%--------------------------------------------------------------------------
\newtheorem{thm}{Teorema}[section]
\newtheorem{cor}[thm]{Corolario}
\newtheorem{lem}[thm]{Lema}
\newtheorem{prop}[thm]{Proposici\'on}
\theoremstyle{definition}
\newtheorem{defn}[thm]{Definici\'on}
\theoremstyle{remark}
\newtheorem{rem}[thm]{Observaci\'on}

% Atajos.
% Se pueden definir comandos nuevos para acortar cosas que se usan
% frecuentemente. Como ejemplo, aqui se definen la R y la Z dobles que
% suelen representar a los conjuntos de numeros reales y enteros.
%--------------------------------------------------------------------------

\def\RR{\mathbb{R}}
\def\ZZ{\mathbb{Z}}

% De la misma forma se pueden definir comandos con argumentos. Por
% ejemplo, aqui definimos un comando para escribir el valor absoluto
% de algo mas facilmente.
%--------------------------------------------------------------------------
\newcommand{\abs}[1]{\left\vert#1\right\vert}

% Operadores.
% Los operadores nuevos deben definirse como tales para que aparezcan
% correctamente. Como ejemplo definimos en jacobiano:
%--------------------------------------------------------------------------
\DeclareMathOperator{\Jac}{Jac}

%--------------------------------------------------------------------------
\title{Actividades sesi\'on 12 de octubre\\
An\'alisis de Algoritmos IV\\
Maestr\'ia en Ciencias de la Computaci\'on\\ 
Oto\~{n}o 2020\\
BUAP
}


\author{Equipo 3\\
  \small Erick Barrios Gonz\'alez\\
  \small Oscar Eduardo Gonz\'alez Ramos\\
  \small Oswaldo Jair Garc\'ia Franco
}

\begin{document}

\maketitle
\clearpage
%\abstract{Analisis de algoritmos}

\section{Ejercicios}
Resuelva las siguientes recurrencias exactamente,primero mediante cambio de variable,
luego verifique sus respuestas empleando el resultado del ejemplo 4.\\

%inicia ejercicio 1.1

\begin{enumerate}[1]	
\item  $$T(n) = \left \{ 
\begin{matrix} 
1 & \mbox{si n = }1\\ 
2T(n/2)+1 & \mbox{en}\mbox{ otro caso}\\
\end{matrix}
\right. $$\\


\begin{center}
Reemplazamos $n$ por $2^{i}$: \\
$t_{i}= T(2^{i})$\\
\vspace{.5cm}
$n/2$ se convierte en :\\
$\frac{2^{i}}{2} =2^{i-1} $\\
\vspace{.5cm}
Entonces:\\
$t_{i}= T(2^{i})= 2T(2^{i-1})+1$\\
\hspace{1.6cm}$=2t_{i-1}+1 $\\
\vspace{.5cm}
Se reescribe como:\\
$t_{i}- 2t_{i-1} = 1$\\
\vspace{.5cm}
Su polinomio caracter\'istico es:\\
$(x-2)(x-1)$\\
\vspace{.5cm}
As\'i todas las soluciones para $t_{i}$ son de la forma:\\
$t_{i} = c_{1}(2^{i}) + c_{2}$\\
\vspace{.5cm}
Ahora regresamos a la ecuaci\'on original\\
$T(n)= c_{1}2^{lgn}+ c2$ $=$ $c_{1}n + c_{2}$ $(1)$\\
\vspace{.5cm}
Como queremos obtener el orden exacto debemos probar que $c_{1}$ es
estrictamente positivo.\\
\vspace{.5cm}
Sustituyendo la soluci\'on proporcionada por la ecuaci\'on (1) en la recurencia original.\\
$1 = T(n)-2T(n/2)$\\
\hspace{1.6cm}$= (c_{1}n+c2)- 2(c_{1}\frac{n}{2} + c_{2})$\\
$c_{2}= -1$\\
No obtuvimos el valor de $c_{1}$ pero estamos en condiciones de afirmar que 
debe ser estrictamente positiva, de lo contrario la ecuaci\'on indicar\'ia falsamente que $T(n)$ es negativa.\\
Entonces queda establecido que:\\
$T(n) \in \Theta(n|n$ es potencia de 2)\\

\vspace{.5cm}
Comprobaci\'on con teorema Maestro:
$l = 2$, $b = 2$,$k = 0$\\
$l > 2^{0} = 1$\\
Entonces $T(n) \in \Theta(n^{\log_{2}(2)})$\\
$T(n) \in \Theta(n|n$ es potencia de 2)

\end{center}
\clearpage
%termina ejercicio 1.1

%inicio del ejercicio 1.2
\item  $$T(n) = \left \{ 
\begin{matrix} 
1 & \mbox{si n = }1\\ 
4T(n/2)+n & \mbox{en}\mbox{ otro caso}\\
\end{matrix}
\right. $$\\


\begin{center}
Reemplazamos $n$ por $2^{i}$: \\
$t_{i}= T(2^{i})$\\
\vspace{.5cm}
$n/2$ se convierte en :\\
$\frac{2^{i}}{2} =2^{i-1} $\\
\vspace{.5cm}
Entonces:\\
$t_{i}= T(2^{i})= 4T(2^{i-1})+ 2^{i}$\\
\hspace{1.6cm}$=4t_{i-1}+2^{i} $\\
\vspace{.5cm}
Se reescribe como:\\
$t_{i}- 4t_{i-1} = 2^{i}$\\
\vspace{.5cm}
Su polinomio caracter\'istico es:\\
$(x-4)(x-2)$\\
\vspace{.5cm}
As\'i todas las soluciones para $t_{i}$ son de la forma:\\
$t_{i} = c_{1}(4^{i}) + c_{2}(2^{i})$\\
\vspace{.5cm}
Ahora regresamos a la ecuaci\'on original\\
$T(n)= c_{1}4^{lgn}+ c_{2}2^{lgn}$ $=$ $c_{1}n^{2} + c_{2}n$ $(1)$\\
\vspace{.5cm}
Como queremos obtener el orden exacto debemos probar que $c_{1}$ es
estrictamente positivo.\\
Sustituyendo la soluci\'on proporcionada por la ecuaci\'on (1) en la recurencia original.\\
$n = T(n)-4T(n/2)$\\
\hspace{1.6cm}$= (c_{1}n^{lg4} + c_{2}n)- 4(c_{1}(\frac{n}{2})^{lg4} + c_{2}(\frac{n}{2}))$\\
$c_{2}= -1$\\
No obtuvimos el valor de $c_{1}$ pero estamos en condiciones de afirmar que 
debe ser estrictamente positiva, de lo contrario la ecuaci\'on  indicar\'ia falsamente
que $T(n)$ es negativa.\\

Entonces queda establecido que:\\
$T(n) \in \Theta(n^{2}|n$ es potencia de 2)\\
\vspace{.5cm}
Comprobaci\'on con teorema Maestro:
$l = 4$, $b = 2$,$k = 0$\\
$l > 2^{0} = 1$\\
Entonces $T(n) \in \Theta(n^{\log_{2}(4)})$\\
$T(n) \in \Theta(n^{2}|n$ es potencia de 2)

\end{center}
\clearpage
%fin ejercicio 1.2


%inicio del ejercicio 1.3
\item  $$T(n) = \left \{ 
\begin{matrix} 
1 & \mbox{si n = }1\\ 
2T(n/2)+ lgn & \mbox{en}\mbox{ otro caso}\\
\end{matrix}
\right. $$\\


\begin{center}
Reemplazamos $n$ por $2^{i}$: \\
$t_{i}= T(2^{i})$\\
\vspace{.5cm}
$n/2$ se convierte en :\\
$\frac{2^{i}}{2} =2^{i-1} $\\
\vspace{.5cm}
Entonces:\\
$t_{i}= T(2^{i})=2T(2^{i-1})+ lg(2^{i})$\\
\hspace{1.6cm}$=2t_{i-1}+ i lg2 $\\
\hspace{1.6cm}$=2t_{i-1}+ i $\\
\vspace{.5cm}
Se reescribe como:\\
$t_{i}- 2t_{i-1} = i $\\
\vspace{.5cm}
Su polinomio caracter\'istico es:\\
$(x-2)(x-1)$\\
\vspace{.5cm}
As\'i todas las soluciones para $t_{i}$ son de la forma:\\
$t_{i} = c_{1} 2^i + c_{2}  + c_3  i $\\
\vspace{.5cm}
Ahora regresamos a la ecuaci\'on original\\
$T(n)= c_{1}n + c_{2} $\\
\vspace{.5cm}
Como queremos obtener el orden exacto debemos probar que $c_{1}$ es
estrictamente positivo.\\
Sustituyendo la soluci\'on proporcionada por la ecuaci\'on en la recurencia original.\\
$lgn = T(n)-2T(n/2)$\\
\hspace{1.6cm}$= (c_{1}n + c_{2})- 2(c_{1}(\frac{n}{2}) + c_{2})$\\
$c_{2}= - log n$\\
No obtuvimos el valor de $c_{1}$ pero estamos en condiciones de afirmar que 
debe ser estrictamente positiva
Por lo tanto :\\
$T(n) \in \Theta(n | n$ es potencia de 2)\\
\vspace{.5cm}
Comprobaci\'on con teorema Maestro:
$l = 2$, $b = 2$,$k = 0$\\
$l > b^{k}$\\
Porque: $2 > 2^{0}$\\
Por lo tanto:\\
$T(n) \in \Theta(n^{\log_{2}(2)})$\\
$T(n) \in \Theta(n|n$ es potencia de 2)

\end{center}
\clearpage
%fin ejercicio 1.3


%inicio del ejercicio 1.4
\item  $$T(n) = \left \{ 
\begin{matrix} 
1 & \mbox{si n = }1\\ 
5T(n/2)+ (n lgn)^2 & \mbox{en}\mbox{ otro caso}\\
\end{matrix}
\right. $$\\


\begin{center}
Reemplazamos $n$ por $2^{i}$: \\
$t_{i}= T(2^{i})$\\
\vspace{.5cm}
$n/2$ se convierte en :\\
$\frac{2^{i}}{2} =2^{i-1} $\\
\vspace{.5cm}
Entonces:\\
$t_{i}= T(2^{i})=5T(2^{i-1})+(2^{i} lg(2^{i}))^2$\\
\hspace{1.6cm}$= 5T(2^{i-1})+(2^{i} i)^2$\\
\vspace{.5cm}
Se reescribe como:\\
$t_{i}- 5t_{i-1} = i^2 4^i $\\
\vspace{.5cm}
Su polinomio caracter\'istico es:\\
$(x-5)(x-4)$\\
\vspace{.5cm}
As\'i todas las soluciones para $t_{i}$ son de la forma:\\
$t_{i} = c_{1} 5^i + c_{2}  4^i $\\
\vspace{.5cm}
Ahora regresamos a la ecuaci\'on original\\
$T(n)= c_{1}n^{lg5} + c_{2} n^{lg4} $\\
\vspace{.5cm}
Como queremos obtener el orden exacto debemos probar que $c_{1}$ es
estrictamente positivo.\\
Sustituyendo la soluci\'on proporcionada por la ecuaci\'on  en la recurencia original.\\
$(nlgn)^2 = T(n)-5T(n/2)$\\
\hspace{1.6cm}$= (c_{1}n^{lg5} + c_{2}n^{lg4})- 2(c_{1}(\frac{n}{2})^{lg5} + c_{2} (\frac{n}{2})^{lg4} )$\\
$c_{2}n^{lg4} = - (n log n)^2 / 4$\\
Como $ c_{2}$ es negativa estamos en condiciones de afirmar que $c_{1}$
debe ser estrictamente positiva
Por lo tanto :\\
$T(n) \in O(n^{log5} | n$ es potencia de 2)\\
\vspace{.5cm}
Comprobaci\'on con teorema Maestro:
$l = 5$, $b = 2$,$k = 0$\\
$l > b^{k}$\\
Porque: $5 > 2^{0}$\\
Por lo tanto:\\
$T(n) \in \Theta(n^{\log_{2}(5)})$\\
$T(n) \in \Theta(n^{\log_{2}(5)} |n$ es potencia de 2)

\end{center}
\clearpage
%fin ejercicio 1.4

%Inicio ejercicio 2.1
\item  $$T(n) = \left \{ 
\begin{matrix} 
1 & \mbox{si n = }1\\ 
2T(n/3)+ 4 & \mbox{en}\mbox{ otro caso}\\
\end{matrix}
\right. $$\\


\begin{center}
Reemplazamos $n$ por $3^{i}$: \\
$t_{i}= T(3^{i})$\\
$t_{i}= T(3^{i})=2T(3^{i-1})+4$\\
\vspace{.5cm}
Se reescribe como:\\
$t_{i}- 2t_{i-1} = 4 $\\
\vspace{.5cm}
Su polinomio caracter\'istico es:\\
$(x-2)(x-4)$\\
\vspace{.5cm}
As\'i todas las soluciones para $t_{i}$ son de la forma:\\
$t_{i} = c_{1} 2^i + c_{2}  4^i $\\
\vspace{.5cm}
Ahora regresamos a la ecuaci\'on original\\
$T(n) = C_{1}2^{lgn}+C_{2}4^{lgn}$\\
$T(n) = C_{1}n^{lg2}+C_{2}n^{lg4}$ (1)\\
\vspace{.5cm}
Como queremos obtener el orden exacto debemos probar que $c_{1}$ es estrictamente positivo.\\
Sustituyendo la soluci\'on proporcionada por la ecuaci\'on en la recurencia original.\\
$4 = T(n)-2T(n/3)$\\
\hspace{1.6cm}$4 = (c_{1}n^{lg2} + c_{2}n^{lg4})- 2(c_{1}(\frac{n}{3})^{lg2} + c_{2} (\frac{n}{2})^{lg4} )$\\
$c_2=-(\frac{8}{n^{lg4}}) $\\

Por lo tanto se puede concluir que :\\
$T(n) \in  \Theta(n^{lg2} | n$ es potencia de 3)\\
\vspace{.5cm}
Comprobaci\'on con teorema Maestro:
$l = 2$, $b = 3$,$k = 0$\\
$l > b^{k}$\\
Porque: $2 > 3^{0}$\\
Por lo tanto:\\
$T(n) \in  \Theta(n^{\log_{3}(2)})$\\
$T(n) \in  \Theta(n^{\log_{3}(2)} |n$ es potencia de 3)

\end{center}
\clearpage
%Fin ejercicio 2.1

%Inicio ejercicio 2.2
\item  $$T(n) = \left \{ 
\begin{matrix} 
1 & \mbox{si n = }1\\ 
T(n/3)+ 2 & \mbox{en}\mbox{ otro caso}\\
\end{matrix}
\right. $$\\


\begin{center}
Reemplazamos $n$ por $3^{i}$: \\
$t_{i}= T(3^{i})$\\
$t_{i}= T(3^{i})=T(3^{i-1})+2$\\
Con manipulaci\'on algebraica obtenemos:\\
$t_{i}-2t_{i-1}+t_{i-2}=0$\\
\vspace{.5cm}
Su polinomio caracter\'istico es:\\
$(x-1)^2$\\
\vspace{.5cm}
As\'i todas las soluciones para $t_{i}$ son de la forma:\\
$t_{i} = c_{1} 1 + c_{2}  i $\\
\vspace{.5cm}
Ahora regresamos a la ecuaci\'on original\\
$T(n)=C_{1}1^{lgn} + C_{2}lgn = C_{1}+ C_{2}lgn$  \\
\vspace{.5cm}
Por lo tanto se puede concluir que :\\
$T(n) \in O(lgn | n$ es potencia de 3)\\
\vspace{.5cm}
Comprobaci\'on con teorema Maestro:
$l = 1$, $b = 3$,$k = 0$\\
$l = b^{k}$\\
Porque: $1 = 3^{0}$\\
Por lo tanto:\\
$T(n) \in \Theta(n^{0}\log n) = \Theta(\log n)$\\

\end{center}
\clearpage
%fin ejercicio 2.2

%Inicio ejercicio 3
\item  $$T(n) = \left \{ 
\begin{matrix} 
1 & \mbox{si n = }2\\ 
2T(\sqrt{n})+ lgn & \mbox{en}\mbox{ otro caso}\\
\end{matrix}
\right. $$\\


\begin{center}
Reemplazamos $n$ por $2^{i}$: \\
$t_{i}= T(2^{i})$\\
$t_{i}= T(2^{i})=2T(2^{\frac{i}{2}})+i$\\
$S(\frac{i}{2})= T(2^{\frac{1}{2}})$\\
Siendo la forma que podemos resolver:\\
Sustituimos $i$ por $2^{i}$\\
$Si=2S(2^{i-1})+i$\\
$Si-2Si-1=i$\\
\vspace{.5cm}
\vspace{.5cm}
Su polinomio caracter\'istico es:\\
$(x-2)(x-1)^2$\\
$S(2^{i})=Si$\\
\vspace{.5cm}
Con soluciones de la forma:\\
$S(n)=C_{1}2^{i} + C_{2} + C_{3}i$\\
$T(n)=\log n \log \log n$\\
\vspace{.5cm}
Ahora regresamos a la ecuaci\'on original\\
$T(n)=C_{1}2^{\log n \log \log n} + C_{2} + C_{3} \log n \log \log n $\\
$T(n)=C_{1}n \log \log n + C_{2} + C_{3} \log n \log (\log n) $\\
\vspace{.5cm}


Por lo tanto se puede concluir que :\\
$T(n) \in O(\log n \log (\log n))$\\
\vspace{.5cm}
Comprobaci\'on con teorema Maestro:\\
$T(n) = 2T(\sqrt{n})+\log n$\\
$i = \log n$ producciones\\
$T(2^i)=2T(2^{\frac{i}{2}})+i$\\
Luego $S(i) = T(2^i)$ produce la recurrencia\\
$S(i)=2S(\frac{i}{2})+i$\\
Entonces $T(n) = T(2^i) = S(i) =  \Theta(i \log i) =  \Theta(\log n (\log \log n))$
\end{center}
\clearpage
%fin ejercicio 3


\item \textbf{Teorema Maestro}\\
El teorema maestro proporciona una manera de resolver recurrencias de la forma:\\

$$T(n) = \left \{ 
\begin{matrix} 
f(n) & \mbox{si 0}  \leq n \leq c\\ 
aT(\frac{n}{b})+ bn^k & \mbox{si c } \leq n\\
\end{matrix} 
\right. $$\\

$$T_2(n) = \left \{ 
\begin{matrix} 
f(n) & \mbox{si 0}  \leq n \leq c\\ 
aT(n-c)+ bn^k & \mbox{si c } \leq n\\
\end{matrix} 
\right. $$\\

$$T(n) \in \left \{ 
\begin{matrix} 
\Theta (n^k)& \mbox{si a}  < 1\\ 
\Theta (n^{k+1})& \mbox{si a}  = 1\\ 
\Theta (n^{a/c})& \mbox{si a} >  1\\ 
\end{matrix} 
\right. $$\\

$$T_2(n) \in \left \{ 
\begin{matrix} 
\Theta (n^k)& \mbox{si a}  < c^k\\ 
\Theta (n^{k} logn)& \mbox{si a}  = c^k\\ 
\Theta (n^{log_c a})& \mbox{si a} >  c^k\\ 
\end{matrix} 
\right. $$\\


Para el caso que usamos durante los ejercicios donde $a\geq1$ y $b>1$ son constantes y $f(n)$ es una funci\'on asint\'oticamente positiva.\\
Est\'a recurrencia describe el tiempo de un algoritmo que:\\
\begin{itemize}
	\item Divide un problema de tama\~no $n$ en $a$ subproblemas, cada uno de tama\~no $\frac{n}{b}$, donde $a$ y $b$ son constantes positivas.\\
	\item Los $a$ subproblemas se resuelven recursivamente, cada uno en un tiempo $T(\frac{n}{b})$.\\
	\item El costo de dividir el problema y combinar los resultados de los subproblemas est\'a descrito por la funci\'on $f(n)$.\\
\end{itemize}

Entonces $T(n)$ puede estar acotada asint\'oticamente de la forma siguiente:\\
3 casos:\\
\begin{enumerate}[1.]
	\item Si $f(n)= O(n^{\log_{b}{a-\epsilon}})$ para alguna constante $\epsilon>0$ entonces\\
	$$T(n)=\Theta(n^{\log_{b}{a}})$$
	\item Si $f(n)=\Theta(n^{\log_{b}{a}})$, entonces: \\
	$$ T(n)= \Theta(n^{\log_{b}{a}}\log{n})  $$
	\item Si $f(n) = \Omega(n^{\log_{b}{a+\epsilon}})$ para alguna constante $\epsilon>0$, y si $af(\frac{n}{b})\leq cf(n)$ para 	alguna constante $c<1$ y una $n$ suficientemente grande, entonces:
	$$T(n)=\Theta(f(n))$$
\end{enumerate}
En los tres casos se compara la funci\'on $f(n)$ con la funci\'on $n^{\log_{b}{a}}$\\

Intuitivamente, la soluci\'on a la recurrencia estar\'a determinada por la mayor de las funciones.\\
\begin{itemize}
\item En 1. La funci\'on $n^{\log_{b}{a}}$ crece m\'as r\'apido polinomialmente que $f(n)$ por un factor $n^{\epsilon}$. Adem\'as, $f(n)$ satisface la condici\'on de regularidad que $f(n)$ por un factor $n^{\epsilon}$. La soluci\'on esta dada entonces por $n^{\log_{b}{a}}$\\
\item En 3. La funci\'on $f(n)$ crece polinomialmente m\'as r\'apido que $n^{\log_{b}{a}}$ por un factor $n^{\epsilon}$. Adem\'as $f(n)$ satisface la condici\'on de regularidad que dice $af(\frac{n}{b})\leq cf(n)$ para alguna constante $c>1$.\\
\item Los tres casos no cubren todas las posibilidades, el m\'etodo no puede aplicarse si las funciones no crecen polinomialmente m\'as r\'apido o si la condici\'on de regularidad no se cumple.
\end{itemize}

\end{enumerate}

\textbf{Conclusiones}\\
\center
Concluimos que en los ejercicios realizados solamemte utilizamos la forma $T(n)=aT(\frac{n}{b})+f(n)$, de igual forma el teorema maestro sirvio para poder comprobar todas las ecuaciones realizadas, por los m\'etodos vistos anteriormente.
\endcenter



\clearpage

\pagebreak 
% Bibliograf\'ia.
%-----------------------------------------------------------------
\begin{thebibliography}{99}
\bibitem{Mauricio2016} Lavalle Mart\'inez, Jos\'e de Jes\'us; La presentaci\'on sobre la tercera
parte de An\'alisis de algoritmos, An\'alisis y Dise\~{n}o de Algoritmos.
Buap, Oto\~{n}o 2020.

\bibitem {notas} Jim\'enez Salazer, H\'ector y Lavalle Mart\'inez, Jos\'e de Jes\'us; An\'alisis y Dise\~{n}o de Algoritmos. Traducci\'on de partes del libro Fundamentals of Algorithmics de Brassard and Bratley FCC - BUAP, 2020.

\bibitem  {notas}Arturo D\'iaz P\'erez. (2005). Teorema Maestro. 16/10/2020, de CINVESTAV Sitio web: http://delta.cs.cinvestav.mx/~adiaz/anadis/TeoremaMaestro.pdf

\bibitem  {notas} Alonso Jimen\'ez, J. A. (2019). An\'alisis de la complejidad de los algoritmos. Obtenido de Dpto. de Ciencias de la Computaci\'on e Inteligencia Artificial: http://www.cs.us.es/~jalonso/cursos/i1m/temas/tema-28.html


\end{thebibliography}

\end{document}